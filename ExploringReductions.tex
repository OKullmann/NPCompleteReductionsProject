\documentclass[a4paper]{report}
\usepackage{mathtools}
\usepackage{graphicx}
\usepackage{verbatim} 
\scrollmode
%
\usepackage{amsmath}
\usepackage{amsfonts}
\usepackage{amssymb}
\usepackage{latexsym}
\usepackage{stmaryrd}
\usepackage{array}
\usepackage{exscale}

\usepackage[driverfallback=hypertex]{hyperref}


\setcounter{secnumdepth}{2}
\setcounter{tocdepth}{1}


\usepackage{theorem} % am 18.5.2001 von ``Basis'' hierhin verschoben

\renewcommand{\thefootnote}{\arabic{footnote})}%

\newtheorem{defi}{Definition}[section]
\newtheorem{lem}[defi]{Lemma}
\newtheorem{thm}[defi]{Theorem}
\newtheorem{corol}[defi]{Corollary}
\newtheorem{propo}[defi]{Proposition}
\newtheorem{exerc}[defi]{Exercise}
\newtheorem{conj}[defi]{Conjecture}
\newtheorem{examp}[defi]{Example}
\newtheorem{quest}[defi]{Question}
\newtheorem{spec}[defi]{Speculation}
\newtheorem{oprbl}[defi]{Open Problem}


\begin{document}

\title{Reductions for NP-complete problems:\\ From SAT to N-Queens-Completion}

\author{Luke Bevan John\\
  Computer Science Department\\
  Swansea University\\
  Swansea, SA1 8EN, UK
}

\maketitle

\begin{abstract}
  Investigation into reductions from SAT to $N$-Queens completion.
\end{abstract}

\tableofcontents


\setcounter{chapter}{-1}

\chapter{TODOS etc.}
\label{cha:todos}

\section{On Latex}
\label{sec:todoslatex}

\cite{lamport94}


\section{Todos}
\label{sec:todostodos}

\begin{enumerate}
\item As given by OK:
  \begin{enumerate}
  \item Complete Definition \ref{def:CNF}.
  \item Complete Definition \ref{def:CLS}.
  \item Complete Example \ref{exp:CLS}.
  \item Provide a formal definition of ``complexity class'' (Definition \ref{def:complexityclass}).
  \item Write Definition \ref{def:inNP} (NP).
  \item Provide Definition \ref{def:NPcomplete}.
  \end{enumerate}
\item Consolidate Git-repositories.
\item Next meeting: looking at the C++ code.
  \begin{enumerate}
  \item Testing.
  \item Documentation.
  \item Possibly a Solver
  \end{enumerate}
\item Research: improving the reduction SAT to 3SAT:
  \begin{enumerate}
  \item Examples (complete) and written out.
  \item Initial concepts.
  \item Literature overview.
  \item Literature search about the topic of reductions SAT to 3-SAT.
  \item For every relevant piece of literature found, write one paragraph, what's in that piece about relevant to our topic.
  \end{enumerate}
\item Oliver Email

  Just to summarise the discussion we had:
  \begin{enumerate}
     \item We will focus on the reductions, their implementations and experimental evaluation.
    \item With the implementations, C++ will be learned.
    \item The experimental evaluation can be extended to include the use the machine-learning tools for automatic configuration of the reductions.
    \item Start providing basic definitions.
  \end{enumerate}
\end{enumerate}



\section{Plan}
\label{sec:Plan}

\begin{enumerate}
\item Add Karp paper (\cite{Karp1972NP}) to bibtex-file.
\item Write paragraph on paper.\\
definition of nondeterministic algorithm: when an algorthim is presented with a decision problem or more accurately a binary tree. the algorithm has the ability to create copies of itself so that it may pursure several avenues of options of the tree simultaneously and for every node that presents a choice the algorithum can copy itself, as can be clearly seen this can lead to an exponetial growth. if the algorithum finds a path that satisfies the formula the input is accepted. \cite{Karp1972NP}
\end{enumerate}



\chapter{Introduction}
\label{cha:Introduction}

\section{SAT}
\label{sec:BackgroundSAT}

\begin{defi}\label{def:CNF}
  A \textbf{CNF} (``CNF formula'')  XXX  conjunctive normal form formula is a conjunction of clauses, whereby each clause is a disjunction of literals.
\end{defi}

\begin{defi}\label{def:CLS}
  A \textbf{clause-set}  XXX a clause-set is a set with clauses as it's elements.
\end{defi}

\begin{examp}\label{exp:CLS}
  XXX examples for CNF and clause-sets\\
F is CNF formula.\\ 
$F = \bigwedge_{j=1}^{n} c_j = c_1 \wedge c_2 \wedge \dots \wedge c_n$,
$c \in C$,  C is a clause-set.\\
$c_i = \bigvee_{j=1}^{s} x_{ij} =  (x_{i1} \vee x_{i2} \vee \dots \vee x_{is})$\\
$x_{ij}$ is an literal or negation of an literal.
$x \in X$. X is a set of literals.\\

there are various versions of CNF formulas, such as SAT, 3SAT, 1-in-3SAT, 2SAT.
\end{examp}


\section{Complexity theory}
\label{sec:basicscomplexitytheory}

Intuitively, a ``complexity class'' is a set of problems that requires a ``similar amount of resources'' to solve, such as time required. Formally however one needs to be as liberal as possible, since this notion is very basic, while ``resources'' etc.\ is very hard to define:
\begin{defi}\label{def:complexityclass}
  XXX
\end{defi}

Intuitively, a complexity class is said to be ``in NP'' if it is a decision problem, where for the ``yes-answer'' a short witness can be provided, which can be verified in polynomial time. Formally the complexity class ``NP'' is defined as follows:
\begin{defi}\label{def:inNP}
  XXX
\end{defi}

Intuitively, a complexity class is called ``NP-complete'', if it is in NP (see Definition \ref{def:inNP}), and is at least as hard as the hardest NP problems (known as being NP-hard) --- this can be formalised as the property that every \emph{other} problem in NP can be reduced to it:
\begin{defi}\label{def:NPcomplete}
  XXX
\end{defi}


\section{Computational problems}
\label{sec:computationalproblems}

\begin{defi}\label{def:SATproblem}
  The \textbf{SAT problem} XXX also known as Boolean satisfiability problem, does there exist a truth assignment to the literals of the CNF formula in question, such that the CNF formula is determined to be true or satisfied. It should be noted that, this is equivalent weather there is an assignment to the literals of a disjunctive normal form(DNF) formula which results in a tautology(\cite{Karp1972NP}).
\end{defi}

\begin{defi}\label{def:3SATproblem}
  The \textbf{3-SAT problem} XXX does a 3-CNF formulae that contains at most 3 literals per clause have a truth assignment to the literals of the 3-CNF formula, such that the formula is
determined to be true or satisfied. 
\end{defi}




\chapter{SAT to 3SAT}
\label{cha:sat13}


\section{Overview}
\label{sec:sat13Overview}

\begin{enumerate}
\item Overview on \cite{Cook1971NP}:
  \begin{enumerate}
  \item Theorem 1 proves that every language in co-NP can be reduced to TAUT.
  \item The proof of Theorem 1 actually shows that SAT (for CNF) is NP-complete.
  \item Theorem 2 shows that 3-CNF (every clause has length at most 3) is NP-complete. This happens by splitting up a clause $x_1 \vee \dots \vee x_s$ into
    \begin{displaymath}
      v \vee x_1 \vee x_2, \quad \neg v \vee x_3 \vee \dots \vee x_s.
    \end{displaymath}
  \item This all in a different language.
  \item For the proof see Subsection \ref{sec:3SATNPcomp}
  \end{enumerate}
\end{enumerate}



\section{NP-completeness of 3-SAT}
\label{sec:3SATNPcomp}

We assume:
\begin{thm}[{{\cite[Theorem 1]{Cook1971NP}}}]\label{thm:SATNPcomplete}
  The SAT problem (see Definition \ref{def:SATproblem}) is NP-complete (see Definition \ref{def:NPcomplete}).
\end{thm}

The main task of this section is to give a good understanding of the NP-completeness of 3-SAT:
\begin{thm}[{{\cite[Theorem 2]{Cook1971NP}}}]\label{thm:SATNPcomplete}
  The 3-SAT problem (see Definition \ref{def:3SATproblem}) is NP-complete.
\end{thm}





\section{3-SAT to 1-in-3-SAT}
\label{sec:3satto13}



\bibliographystyle{plainurl}
\bibliography{Bibliography}

\end{document}