\documentclass[a4paper]{report}

\setcounter{secnumdepth}{2}
\setcounter{tocdepth}{1}


\begin{document}

\title{Reductions for NP-complete problems}

\author{Luke Bevan John\\
  Computer Science Department\\
  Swansea University\\
  Swansea, SA1 8EN, UK
}
\date{}

\maketitle

\begin{abstract}
  Investigation into reductions from SAT to $N$-Queens completion.
\end{abstract}

\tableofcontents


\setcounter{chapter}{-1}

\chapter{TODOS etc.}
\label{cha:todos}

\section{On Latex}
\label{sec:todoslatex}

\cite{lamport94}


\section{Todos}
\label{sec:todostodos}

\begin{enumerate}
\item Consolidate Git-repositories.
\item Next meeting: looking at the C++ code.
  \begin{enumerate}
  \item Testing.
  \item Documentation.
  \item Possibly a Solver
  \end{enumerate}
\item Research: improving the reduction SAT to 3SAT:
  \begin{enumerate}
  \item Examples (complete).
  \item Examples (complete) and Written out.
  \item Initial concepts.
  \item Literature overview.
  \end{enumerate}
\item Oliver Email\\
 just to summarise the discussion we had:
  \begin{enumerate}
     \item We will focus on the reductions, their implementations and experimental evaluation.
    \item With the implementations, C++ will be learned.
    \item The experimental evaluation can be extended to include the use the machine-learning tools for automatic configuration of the reductions.\\ \\ Oliver
  \end{enumerate}
\end{enumerate}




\chapter {Introduction}
\label{cha:Introduction}

\section{Background}
\label{sec:Background}



\chapter{SAT to 3SAT}
\label{cha:sat13}

\section{3SAT to 1-in-3-SAT}
\label{sec:3satto13}



\bibliographystyle{plainurl}
\bibliography{Bibliography}

\end{document}