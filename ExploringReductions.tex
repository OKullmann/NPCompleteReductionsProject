\documentclass[a4paper]{report}
\usepackage{mathtools}
\usepackage{graphicx}
\usepackage{verbatim} 
\scrollmode
%
\usepackage{amsmath}
\usepackage{amsfonts}
\usepackage{amssymb}
\usepackage{latexsym}
\usepackage{stmaryrd}
\usepackage{array}
\usepackage{exscale}

\usepackage[driverfallback=hypertex]{hyperref}


\setcounter{secnumdepth}{2}
\setcounter{tocdepth}{1}


\usepackage{theorem} % am 18.5.2001 von ``Basis'' hierhin verschoben

\renewcommand{\thefootnote}{\arabic{footnote})}%

\newtheorem{defi}{Definition}[section]
\newtheorem{lem}[defi]{Lemma}
\newtheorem{thm}[defi]{Theorem}
\newtheorem{corol}[defi]{Corollary}
\newtheorem{propo}[defi]{Proposition}
\newtheorem{exerc}[defi]{Exercise}
\newtheorem{conj}[defi]{Conjecture}
\newtheorem{examp}[defi]{Example}
\newtheorem{quest}[defi]{Question}
\newtheorem{spec}[defi]{Speculation}
\newtheorem{oprbl}[defi]{Open Problem}


\begin{document}

\title{Reductions for NP-complete problems:\\ From SAT to N-Queens-Completion}

\author{Luke Bevan John\\
  Computer Science Department\\
  Swansea University\\
  Swansea, SA1 8EN, UK
}

\maketitle

\begin{abstract}
  Investigation into reductions from SAT to $N$-Queens completion.
\end{abstract}

\tableofcontents


\setcounter{chapter}{-1}

\chapter{TODOS etc.}
\label{cha:todos}

\section{On Latex}
\label{sec:todoslatex}

\cite{lamport94}


\section{Todos}
\label{sec:todostodos}

\begin{enumerate}
\item Consolidate Git-repositories.
\item Next meeting: looking at the C++ code.
  \begin{enumerate}
  \item Testing.
  \item Documentation.
  \item Possibly a Solver
  \end{enumerate}
\item Research: improving the reduction SAT to 3SAT:
  \begin{enumerate}
  \item Examples (complete) and written out.
  \item Initial concepts.
  \item Literature overview.
  \item Literature search about the topic of reductions SAT to 3-SAT.
  \item For every relevant piece of literature found, write one paragraph, what's in that piece about relevant to our topic.
  \end{enumerate}
\item Oliver Email

  Just to summarise the discussion we had:
  \begin{enumerate}
     \item We will focus on the reductions, their implementations and experimental evaluation.
    \item With the implementations, C++ will be learned.
    \item The experimental evaluation can be extended to include the use the machine-learning tools for automatic configuration of the reductions.
    \item Start providing basic definitions.
  \end{enumerate}
\end{enumerate}



\section{Plan}
\label{sec:Plan}

\begin{enumerate}
\item Add Karp paper (\cite{Karp1972NP}) to bibtex-file.
\item Write paragraph on paper.
\end{enumerate}



\chapter {Introduction}
\label{cha:Introduction}

\section{Background}
\label{sec:Background}



\chapter{SAT to 3SAT}
\label{cha:sat13}


\section{Overview}
\label{sec:sat13Overview}

\begin{enumerate}
\item \cite{Cook1971NP}
  \begin{enumerate}
  \item Theorem proves that every language in co-NP can be reduced to TAUT.
  \item The proof of Theorem 1 actually shows that SAT (for CNF) is NP-complete.
  \item Theorem 2 shows that 3-CNF (every clause has length at most 3) is NP-complete. This happens by splitting up a clause $x_1 \vee \dots \vee x_s$ into
    \begin{displaymath}
      v \vee x_1 \vee x_2, \quad \neg v \vee x_3 \vee \dots \vee x_s.
    \end{displaymath}
  \item This all in a different language.
  \item For the proof see Subsection \ref{sec:3SATNPcomp}
  \end{enumerate}
\end{enumerate}



\section{NP-completeness of 3-SAT}
\label{sec:3SATNPcomp}







\section{3SAT to 1-in-3-SAT}
\label{sec:3satto13}



\bibliographystyle{plainurl}
\bibliography{Bibliography}

\end{document}